% https://tex.stackexchange.com/questions/27351/speed-up-beamer-compile-time

\documentclass[xcolor={dvipsnames,table},aspectratio=169]{beamer}
% \includeonlyframes{current}

\graphicspath{{graphics/}}

\title{Reviving the Media Station:\\A Study in Reverse Engineering}
\author{Nathanael Gentry}
\institute{School of Engineering}
\date{April 13, 2021}

\usepackage{listings}
\lstset{basicstyle=\ttfamily, keywordstyle=\bfseries, language=Python, escapeinside={(*}{*)}}

\usepackage{lmodern}

% https://stackoverflow.com/questions/1346512/to-escape-many-in-latex-efficiently
\usepackage{underscore}

\usepackage[T1]{fontenc}
\usepackage{caption}
\usepackage{forest}
\usepackage{tcolorbox}
\usepackage{verbatim}

\newcommand{\hexdump}[3]{
  {\ttfamily\small
  \begin{tabular}{@{}c|*{15}{c@{ }}c|l@{}}
    \input{dumps/#1}
  \end{tabular}}
\source{\cite{#2}, \texttt{#3}}
}

% https://tex.stackexchange.com/questions/5073/making-a-simple-directory-tree
\forestset{%
  default preamble={for tree={%
      font=\ttfamily, %
      grow'=0, %
      child anchor=west, %
      parent anchor=south, %
      anchor=west, %
      calign=first, %
      edge path={ %
        \noexpand\path [draw, \forestoption{edge}] %
        (!u.south west) +(7.5pt,0) |- node[fill,inner sep=1.25pt] {} (.child anchor)\forestoption{edge label}; %
      }, %
      before typesetting nodes={ %
        if n=1 %
        {insert before={[,phantom]}} %
        {} %
      }, %
      fit=band, %
      before computing xy={l=15pt}, %
    } %
  } %
}

% Unused ultimately, but still interesting:
% https://tex.stackexchange.com/questions/569923/conflict-of-resizebox-and-verbatim-mode
\newsavebox{\tablebox}

\setbeamertemplate{section in toc}[sections numbered]
\setbeamertemplate{subsection in toc}[subsections numbered]

% ~~~~~ LIGHT THEME ~~~~~~~~~~~~~~~~~~~~~~~~~~~~~~~
% \usetheme{Berkeley}
% \usecolortheme{whale}

% \definecolor{LUblue}{rgb}{0.03921568627, 0.14509803921, 0.30588235294} % UBC Blue (primary)
% \definecolor{LUred}{rgb}{0.6,0,0} % UBC Grey (secondary)

% \setbeamercolor{palette primary}{bg=LUblue,fg=white}
% \setbeamercolor{palette secondary}{bg=LUblue,fg=white}
% \setbeamercolor{palette tertiary}{bg=LUblue,fg=white}
% \setbeamercolor{palette quaternary}{bg=LUblue,fg=white}
% \setbeamercolor{structure}{fg=LUblue} % itemize, enumerate, etc
% \setbeamercolor{section in toc}{fg=LUblue} % TOC sections

% \usebackgroundtemplate{\includegraphics[width=\paperwidth,height=\paperheight]{solidwhite.jpg}}

% ~~~~~ DARK THEME ~~~~~~~~~~~~~~~~~~~~~~~~~~~~~~~
% https://tex.stackexchange.com/questions/57477/beamer-dark-theme
% \usetheme{Warsaw}

\setbeamercolor{normal text}{fg=white,bg=black!90}
\setbeamercolor{structure}{fg=white}

\setbeamercolor{alerted text}{fg=red!85!black}

\setbeamercolor{item projected}{use=item,fg=black,bg=item.fg!35}

\setbeamercolor*{palette primary}{use=structure,fg=structure.fg}
\setbeamercolor*{palette secondary}{use=structure,fg=structure.fg!95!black}
\setbeamercolor*{palette tertiary}{use=structure,fg=structure.fg!90!black}
\setbeamercolor*{palette quaternary}{use=structure,fg=structure.fg!95!black,bg=black!80}

\setbeamercolor*{framesubtitle}{fg=white}

\setbeamercolor*{block title}{parent=structure,bg=black!60}
\setbeamercolor*{block body}{fg=black,bg=black!10}
\setbeamercolor*{block title alerted}{parent=alerted text,bg=black!15}
\setbeamercolor*{block title example}{parent=example text,bg=black!15}

\usebackgroundtemplate{\includegraphics[width=\paperwidth,height=\paperheight]{demossblue}}

% https://tex.stackexchange.com/questions/411313/using-listings-package-with-beamer-block-environments-for-displaying-source-code
% \usepackage{listings}
% \lstset{numbers=left, numberstyle=\tiny, stepnumber=1,firstnumber=1,
%     numbersep=5pt,language=C,
%     stringstyle=\ttfamily,
%     basicstyle=\footnotesize,
%     showstringspaces=false
% }

% https://tex.stackexchange.com/questions/48473/best-way-to-give-sources-of-images-used-in-a-beamer-presentation
\usepackage[absolute,overlay]{textpos}

\setbeamercolor{framesource}{fg=gray}
\setbeamerfont{framesource}{size=\footnotesize}

\newcommand{\source}[1]{\begin{textblock*}{4cm}(11.5cm,8cm)
    \begin{beamercolorbox}[ht=0.5cm,right]{framesource}
        \usebeamerfont{framesource}\usebeamercolor[fg]{framesource} Source: {#1}
    \end{beamercolorbox}
\end{textblock*}}

% https://tex.stackexchange.com/questions/16793/global-setting-of-spacing-between-items-in-itemize-environment-for-beamer
\let\olditem\item
\renewcommand{\item}{\olditem\vspace{8pt}}

% https://tex.stackexchange.com/questions/262115/bold-serif-headings-in-beamer
\setbeamerfont{frametitle}{series=\bfseries,parent=structure}

% https://www.patrickbaylis.com/blog/2018-10-11-beamer-resizing/
\usepackage{adjustbox}
\makeatletter
\newcommand{\fitimage}[2][\@nil]{
  \begin{figure}
    \begin{adjustbox}{width=0.9\textwidth, totalheight=\textheight-2\baselineskip-2\baselineskip,keepaspectratio}
      \includegraphics{#2}
    \end{adjustbox}
    \def\tmp{#1}%
    \ifx\tmp\@nnil
    \else
    \caption{#1}
    \fi
  \end{figure}
}
\makeatother

% https://statisticaloddsandends.wordpress.com/2019/02/18/beamer-inserting-section-slides-before-each-section/
\AtBeginSection[]
{
    \begin{frame}
        \frametitle{Reverse-Engineering Process}
        \tableofcontents[currentsection]
    \end{frame}
}

\renewcommand{\figurename}{}

\begin{document}

\frame{\titlepage}

\usebackgroundtemplate{\includegraphics[width=\paperwidth,height=\paperheight]{solidbluenolu}}
\frame{}

\begin{frame}
  \fitimage{gsocsun}
  \source{\cite{GoogleSummerCode2020}}
\end{frame}

\begin{frame}
  \fitimage{gsoc}
  \source{\cite{GoogleSummerCode2020}}
\end{frame}

\begin{frame}
  \fitimage{scummvm}
  \source{\cite{ScummVM}}
\end{frame}

\begin{frame}
  \vspace{-2em}
  \begin{figure}
    \includegraphics[scale=0.4]{scummvm}
  \end{figure}
  \vspace{1em}\pause
  \begin{itemize}
    \item Open-source reimplementation of game interpreters\pause
    \item Original game data playable on newer and diverse platforms
  \end{itemize}
  \source{\cite{ScummVM}}
\end{frame}

\begin{frame}
  \fitimage{d4cover}
  \source{\cite{TipsTricksDirector1994}}
\end{frame}

\begin{frame}
\fitimage{candle}
\end{frame}

\begin{frame}
  \fitimage{alien}
\end{frame}

\newcommand{\github}{\begin{frame}
  \fitimage{seuss2}
  \vspace{-1.5em}
  \begin{center}
    \huge \url{https://npjg.github.io}
  \end{center}
\end{frame}}
\github

\begin{frame}
  \fitimage{asblogo}
  \source{\cite{DisneyAnimatedStoryBook2020}}
\end{frame}

\begin{frame}
  \fitimage{msitrans}
  \source{\cite{flurryDisneyAnimatedStoryBook1994}}
\end{frame}

\begin{frame}
\begin{center}
  ``I lost my copy of the source many years ago when a backup drive failed. I was
  hoping to do a similar thing to what ScummVM does...''
\end{center}
\end{frame}

\begin{frame}{Research Question}\pause
  \begin{center}
    \huge
    How can an interactive multimedia format be reverse-engineered from scratch?
  \end{center}
\end{frame}

% To show how I have begun answering this question...

\begin{frame}{Reverse-Engineering Process}
  \tableofcontents
\end{frame}

\begin{frame}{Bytes}\pause
\begin{itemize}
  \item \textbf{Base-10} (decimal) uses 10 digits --\pause{} the numbers 0-9.\\\pause
  \item \textbf{Base-2} (binary) uses 2 digits --\pause{} the bits \texttt{0} and \texttt{1}.\\\pause
  \item \textbf{Base-16} (hexadecimal) uses 16 digits --\pause{} the numbers 0-9\pause{} and
    letters A-F.
\end{itemize}
\end{frame}

\begin{frame}{Bytes}\pause
  \begin{center}
    \large

    A \textbf{byte} is a collection of 8 bits.\\\pause
    Each byte represents a value from $0 = (2^0 - 1)$\pause{} to $255 = (2^8 - 1)$.\pause

    \vspace{1em}
    For example,\pause
    \begin{equation*}
      (10101010)_2\pause \equiv \mathbf{(AA)}_{16}\pause \equiv (170)_{10}.\pause
    \end{equation*}

    Additionally,\pause
    \begin{equation*}
      \mathbf{(AA)_{16}}\pause\equiv  \textbf{\texttt{0xAA}}\pause \equiv  \textbf{\texttt{0xaa}}.
    \end{equation*}
  \end{center}
\end{frame}

\section{Understand file formats}

\begin{frame}{Files}\pause
  \begin{center}
    A \textbf{file} is a collection of bytes united under one name.\\\pause
    Each byte has a unique \textbf{address} -- its position in the file.\pause
  \end{center}

  \begin{figure}
    \hexdump{garage-117-0x00-0x80.0}{garage}{117.cxt}
  \end{figure}

  \source{\cite{flurryDisneyAnimatedStoryBook1994}, \texttt{data/107.cxt}}
\end{frame}

\begin{frame}{Files}
  \begin{figure}
    \hexdump{garage-117-0x00-0x80.1}{garage}{117.cxt}
  \end{figure}

  \begin{center}\pause
    \Large \texttt{RIFF}
  \end{center}
\end{frame}

\begin{frame}{Files}
  \begin{figure}
    \hexdump{garage-117-0x00-0x80.2}{garage}{117.cxt}
  \end{figure}

  \begin{center}\pause
    \Large \texttt{0x09\,c7\,66} bytes
  \end{center}
\end{frame}

\begin{frame}[fragile]{RIFF:\pause{} Resource Interchange File Format}
\end{frame}

\begin{frame}{RIFF: Resource Interchange File Format}
  \fitimage{riffstruct}
  \source{\cite{loomisRIFFFileStructure2001}}
\end{frame}

\begin{frame}[fragile]{RIFF: Resource Interchange File Format}
\begin{lstlisting}[firstnumber=1, label=glabels, xleftmargin=10pt]
typedef struct {
     char ckID[4];         // Unique chunk identifier
     uint32 ckSize;        // Size of field <ckData>
     byte *ckData[ckSize]; // Actual data
} chunk;
\end{lstlisting}
\end{frame}

\begin{frame}{RIFF: Resource Interchange File Format}
  \begin{center}
    In a RIFF file, the chunk IDs are often the only plain-text strings in the file.
  \end{center}\pause
  \centerline{\texttt{\$ strings 100.cxt}}
\end{frame}

\begin{frame}{CD-ROM Structure}
  % https://tex.stackexchange.com/questions/23647/drawing-a-directory-listing-a-la-the-tree-command-in-tikz
  \begin{figure}
    \begin{forest}
      [DALMATIANS
      [DATA
      [BOOT.STM,name=boot]
      [100.CXT,name=startcxt] % Include dots here
      [101.CXT]
      [3983.CXT,name=endcxt]
      [PROFILE.\textunderscore ST]
      ] % Include dots here
      [101\textunderscore ASB.EXE]]
      \pause{}
      \node[draw,line width=1mm,fit={(startcxt) (endcxt)},label={[label
        distance=0.3cm]0:\Large Contexts}] {};
    \end{forest}
  \end{figure}
\end{frame}

\begin{frame}{Context Structure}
  \begin{figure}
    \begin{forest}
      [100.CXT
      [RIFF
      [igod] % Include dots here
      [a001]] % Include dots here
      [RIFF
      [a100]] % Include dots here
      [RIFF
      [a200]]]
    \end{forest}
  \end{figure}
\end{frame}

\section{Parse data structures}

% Don't change the space with highlighting:
% https://tex.stackexchange.com/questions/198931/color-overshoots-when-using-cellcolor-without-intercolumn-space
\newcommand{\imageheader}[1]{
  \begin{figure}
    \hexdump{garage-117-0x2200-0xa0.#1}{garage}{117.cxt}
  \end{figure}
}

\foreach\x in {0,1,2,3,4,5,6,7,8,9,10}{
  \begin{frame}{Data Structures}
    \imageheader{\x}
  \end{frame}
}

% TODO: Replace this with the location of all 03 00 bytes.
% I can implement a basic search parser.
\begin{frame}{Data Structures}
  \begin{figure}
    \hexdump{garage-117-0x2200-0xa0.10}{garage}{117.cxt}
  \end{figure}
  \begin{center}
    Why are there so many \texttt{03 00} bytes?
  \end{center}
\end{frame}

\begin{frame}[fragile]{Data Structures}\pause
\begin{lstlisting}
class DatumType(IntEnum):
    UINT8      = 0x0002,
    UINT16     = 0x0003,
    SINT16     = 0x0010,
    STRING     = 0x0012,
    FILE       = 0x000a,
    POINT      = 0x000f,
    REF        = 0x001b,
    BBOX       = 0x000d,
    POLY       = 0x001d
\end{lstlisting}\pause
  \begin{center}
    \Large \textbf{datum}\pause{} (data):\pause{} a single piece of information.
  \end{center}
\end{frame}

\begin{frame}[fragile]{Data Structures}
\begin{lstlisting}
class DatumType(IntEnum):
    UINT8      = 0x0002,
    UINT16     = 0x0003,
    SINT16     = 0x0010,
    STRING     = 0x0012,
    FILE       = 0x000a,
    (*\LARGE \verb|POINT   = 0x000f|*),
    REF        = 0x001b,
    BBOX       = 0x000d,
    POLY       = 0x001d
\end{lstlisting}
  \begin{center}
    \Large \textbf{datum} (data): a single piece of information.
  \end{center}
\end{frame}

\begin{frame}{Data Structures}
  \begin{center}
    In reverse engineering, the discoveries you make build upon themselves dramatically.
  \end{center}
\end{frame}

\begin{frame}[label=current]{Data Structures}
  \imageheader{0}
  \begin{center}
    Background\pause{} $\implies$\pause{} Full-screen $\implies$\pause{} VGA Graphics $\implies$\pause{}
    $640\times 480$
  \end{center}\pause\vspace{-1.5em}
  \begin{center}
    \LARGE $640\equiv \texttt{0x02\,80} \equiv \texttt{08 20}$
  \end{center}
\end{frame}

\begin{frame}[label=current]{Data Structures}
  \imageheader{11}
  \begin{center}
    Background $\implies$ Full-screen $\implies$ VGA Graphics $\implies$
    $640\times 480$
  \end{center}\vspace{-1.5em}
  \begin{center}
    \LARGE $640\equiv \texttt{0x02\,80} \equiv \texttt{08 20}$
  \end{center}
\end{frame}

\begin{frame}[label=current]{Data Structures}
  \imageheader{12}
  \begin{center}
    Background $\implies$ Full-screen $\implies$ VGA Graphics $\implies$
    $640\times 480$
  \end{center}\vspace{-1.5em}
  \begin{center}
    \LARGE $480\equiv \texttt{0x01\,e0} \equiv \texttt{e0 01}$
  \end{center}
\end{frame}

\begin{frame}[label=current]{Data Structures}
  \imageheader{13}
  \begin{center}
    Background $\implies$ Full-screen $\implies$ VGA Graphics $\implies$
    $640\times 480$
  \end{center}\vspace{-1.5em}\pause
  \begin{center}
    \Large Point: $(x, y)$
  \end{center}
\end{frame}

\begin{frame}[fragile,label=current]{Data Structures}
\begin{figure}
\begin{lstlisting}
class Point(Object):
    def __init__(self, m):
        value_assert(m, b'\x10\x00')
        self.x = struct.unpack("<H", m.read(2))[0]

        value_assert(m, b'\x10\x00')
        self.y = struct.unpack("<H", m.read(2))[0]
\end{lstlisting}
\end{figure}
\end{frame}

\section{Extract game assets}

\section{Write interactive engine}

\begin{frame}[allowframebreaks]{References}
  \bibliographystyle{IEEEtran}
  \bibliography{MediaStation}
\end{frame}
\end{document}
