\documentclass{amsart}

\title{2021 Research Week Submission}
\author{Nathanael Gentry}
\date{\today}

\begin{document}
\maketitle

\begin{enumerate}
\item \textbf{Title}: Reviving the Media Station: A Study in Reverse Engineering
\item \textbf{Program of study}: Computer science (student's major is Mathematics)
\item \textbf{Presentation type}: Oral
\item \textbf{Subtype}: Applied
\item \textbf{Menthor}: Dr. Melesa Poole (mbolt@liberty.edu)
\item \textbf{Student:} Nathanael Gentry (ngentry1@liberty.edu)
\end{enumerate}

\section{Abstract}

In the 1990s, home computers grew powerful enough to display rich multimedia
content. CD-ROM games especially benefited from this technology boom. However,
in the years since the dot-com bust, much of the original source code for these
classic titles has been lost or kept under copyright. Often the multimedia
asset data are kept in platform-independent binary formats, but actually playing
back the content requires running the original interpreters under legacy
operating systems in a virtual machine. However, the ScummVM project
(https://github.com/scummvm) seeks to keep alive historic multimedia content --
particularly point-and-click adventure games -- by reverse-engineering the
interpreters and reimplementing them for modern operating systems. Thus, the
original data can be played natively on a modern computer, with little
virtualization overhead. This presentation describes the \textit{technical
  procedures and results of reverse-engineering from scratch the multimedia
  presentation engine developed by Media Station, Inc.}, a defunct Michigan
media company. Media Station partnered with Disney on several titles in the
Disney animated storybook series, which was a significant multimedia achievement
for Disney. A complete reimplementation of the Media Station engine would add
support for at least 17 titles released from 1994 to 2002. ScummVM would also
finish gaining support for the Living Books series. The presentation will show
how reverse-engineering is a rewarding experience, especially when it helps
preserve such interesting and fun aspects of computer history.

\section{Christian worldview integration}

In essence, all of science is reverse-engineering -- we are discovering the
glories that God has hidden into the world for us to discover. Reversing only
works because someone has engineered order in the first place. Many early
scientists got their motivation from exactly this. Since I have
reverse-engineered a multimedia engine that brought me such joy as a child
through the Disney animated storybooks made with the engine, I am helping bring
a "joyful heart," which is "good medicine" to a new generation of children
(Proverbs 17:22). By bringing the careful eyes of a reverse-engineer to recreate
this engine from scratch, I am also putting "meat on the bones" on my childhood
love of computers. As Paul writes in 1 Corinthians 13, when he was a child he
thought like a child. But when he became a man he put childish things behind him
and looked toward the time when he could be fully grown up -- seeing the world
not through a dim reflection but as it really is. Looking at how I have come to
such a deep understanding of software I used to play as a child reminds me very
much of this growing-up process. Moreover, I am interested in digital forensics,
and all the good digital forensic scientists I know started their detective work
on game engines. In this way, then, this work is as much a preparatory exercise
as it is a complete project in itself: Even though we hope we will live full
lives on the earth, the fullest life on earth is only a preparation for heaven.

\end{document}
